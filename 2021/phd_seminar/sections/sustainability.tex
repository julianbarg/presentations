%!TEX root = ../main.tex

\section{Sustainability \& Learning}
\begin{frame}
	\frametitle{Table of Contents}
	\tableofcontents[currentsection,currentsubsection]
	\note{An analysis of sustainability, using the language of learning, reliability \& validity.}
\end{frame}

\begin{frame}
	Why should we (sustainability researchers) care about reliability \& validity?

	% \hrulefill

	\note{The sustainability literature, read with attention to learning, reliability \& validity.

	Implicit model of learning in the literature.}
\end{frame}

\blackgroup
\begin{frame}[plain]
	\note{
		\tiny\begin{itemize}
			\item \citetitle{Hart1995b} \citep{Hart1995b}
			\item \citetitle{Purser1995} \citep{Purser1995}
			\item \citetitle{Banerjee2003} \citep{Banerjee2003}
			\item \citetitle{Bansal2005} \citep{Bansal2005}
			\item \citetitle{Bansal2014} \citep{Bansal2014}
			\item \citetitle{Hoffman2015} \citep{Hoffman2015}
			\item \citetitle{Ergene2020} \citep{Ergene2020}
	\end{itemize}
	}
\end{frame}
\egroup

\begin{frame}{Sustainability theory I}
	\begin{columns}[t]
		\begin{column}{0.45\textwidth}
			\begin{block}{Validity--\\Environmental management}
				\begin{enumerate}
					\item <1-> Organizational level narratives
					\item <2-> Technology \& clean-up
					\emptyline
					\item <3-> Rationality \& bounded rationality
					\item <4-> Learning diffuses horizontally
				\end{enumerate}
			\end{block}
		\end{column}
		\vline
		\hspace{2pt}
		\begin{column}{0.45\textwidth}
			\begin{block}{Reliability--\\Ecocentrism}
				\begin{enumerate}
					\item <1-> Organizational level and above
					\item <2-> Greenwashing \& pollution
					\item <3-> Social constructivism
					\emptyline
					\item <4-> Learning meets counterforce
				\end{enumerate}
			\end{block}
		\end{column}
	\end{columns}

	\hrulefill

	{\footnote{For now borrowing terminology from \citet{Purser1995}}}

	\emptyline

	\onslide<5->{$\Rightarrow$ Underlying models of change \& collective learning}
	\note{4. Counterforce is power, organizations learning how to live with the rules, e.g., \citet{Wright2017}.}
\end{frame}

\blackgroup
\begin{frame}[plain]
	% Purpose II
	\note{How models on dissemination of learning, models of the world influence research and the findings that we look for.}	
\end{frame}
\egroup